\documentclass[12pt,twoside,a4paper,titlepage]{report}

\setlength{\oddsidemargin}{2mm}
\setlength{\evensidemargin}{2mm}
\setlength{\textwidth}{154mm}
\setlength{\textheight}{220mm}
\setlength{\topmargin}{0mm}
\setlength{\parskip}{1mm}
\renewcommand{\floatpagefraction} {0.99}
\renewcommand{\textfraction} {0.0}
\renewcommand{\topfraction} {0.99}

\renewcommand{\baselinestretch}{1.1}

\vbadness=10001 % the hell with underfull vboxes
\hbadness=10001

%\usepackage[romanian]{babel}
\usepackage[latin2]{inputenc}
\usepackage{fancyhdr}
%\usepackage{makeidx}
\usepackage{amssymb}
\usepackage{url}
\usepackage{longtable}
\usepackage{multicol}
\usepackage{graphics}
\usepackage{graphicx} 
\usepackage{subfigure}
\usepackage{pslatex}
\usepackage{multirow}
\usepackage[colorlinks=true, pdfstartview=FitV, linkcolor=blue, 
	    citecolor=blue, urlcolor=blue]{hyperref}

\pagestyle{fancy}

\addtolength{\headheight}{\baselineskip}

\renewcommand{\chaptermark}[1]{\markboth{\chaptername \ \thechapter.\ #1}{}}
\renewcommand{\sectionmark}[1]{\markright{\thesection.\ #1}}

\newcommand{\clearemptydoublepage}{\newpage{\pagestyle{empty}\cleardoublepage}}

\setcounter{secnumdepth}{5}

\begin{document} 

%\newcommand{\helv}{%
% \fontfamily{phv}\fontseries{b}\fontsize{9}{11}\selectfont}

\fancyhf{}

\fancyhead{} % clear all fields
\fancyhead[LO]{\rightmark }
\fancyhead[RE]{\leftmark }
\fancyhead[LE,RO]{\thepage }
%\fancyfoot[CO,CE]{\thepage }

\pagenumbering{roman}

%\title{automatic firewall management}
%\author{author: Petre Rodan}
%\date{created Nov 19 2015, revised \today}
%\maketitle

%\tableofcontents
%\newpage

\section{laboule - principles of operation}
\pagenumbering{arabic}

the scope is to manage a firewall in an automatic manner based on external events. in case unwanted behaviour is detected and if the magnitude is above a customizable threshold then the daemon will jail that IP for a time period thus not allowing the attacker to flood any of the services present on the server. the denial rule will be automatically removed once that period is over. 

if it happens that the IP is a repeating offender the ban periods will get longer and longer.

the proposed service has two distinct modules: 
\begin{itemize}
    \item a daemon (laboule itself) listening on a unix socket for simple commands
    \item multiple scripts that detect fishy behaviour and send commands to the daemon
\end{itemize}

\section{the laboule daemon}

it listens on a UNIX socket for commands like
\begin{verbatim}
213.233.121.11 authfail count
\end{verbatim}

each time a \verb!count! command is received the incident will be internally logged and counted, later to be compared to a threshold level. once the threshold is reached that IP will be added to a firewall chain whom all incoming packets are filtered by.

all the time intervals and the event counting is done internally.

\verb!ban! commands jail the IP on the spot.

\section{detection scripts}

these can be written in any language. they either exit without taking any further action or simply pipe commands like "212.130.11.20 COUNT " into laboule's socket.

any server generated output (be it log, other tools, proc files) is parsed and if a detection rule is matched then the IP will be reported to laboule as many times as the behaviour was detected. this is an asyncronous approach and it's the prefered method.

if however a syncronous decision needs to be made then a detection script can be directly tied into a syslog filter or can be made to receive input via a pipe in order to parse possible events on the spot.

examples of such scripts can detect any of: 
\begin{itemize}
    \item port scanning attempts
    \item detection of malware that generate http GETs of a single page of a website
    \item floods (based on kernel connection tracking tables)
    \item failed login attempts of any kind
\end{itemize}

\end{document}
